\subsection*{REBEL}
We use REBEL because it is end-to-end (it finds entities and relations simultaneously), open-source, and easy to implement using Hugging Face. Additionally, as per the DocRED paper by Yao et al \cite{yao2019DocRED}, pretrained REBEL currently yields the best joint entity and relation extraction (NER and RE) results compared with the benchmark among all models sampled, achieving a relation F1 score of 47.1\footnote{https://paperswithcode.com/sota/joint-entity-and-relation-extraction-on-3}.


Since the pre-trained REBEL model has a token limit, we split the LD into 128-token chunks before extracting 3 head-relation-tail triplets from each chunk. We split the text into 128 token chunks as it is approximately the length of one paragraph. Through visual inspection, we find that there are typically 3 triples in each paragraph. Moreover, since REBEL employs beam search, the number of triples must be less than or equal to the number of beams. We determine that the optimal number of beams, based on runtime, is 3 beams, which means that the maximum triples per chunk would be 3.


Once the triplets are extracted, we use NetworkX to create a directed graph, and use MatPlotLib to visualize and plot the results. Below is a sample image of a KG produced from a gold summary.


\hl{<Insert image / plot of KG>}


\hl{<Why extract triplets, and not extract triplets typed?>}